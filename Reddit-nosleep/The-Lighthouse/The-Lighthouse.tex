\documentclass[a5paper]{scrartcl}
\usepackage{amssymb, amsmath} % needed for math
\usepackage[utf8]{inputenc} % this is needed for umlauts
\usepackage[USenglish]{babel} % this is needed for umlauts
\usepackage[T1]{fontenc}  % this is needed for correct output of umlauts in pdf
%\usepackage[margin=2.5cm]{geometry} %layout
\usepackage{hyperref}  % links im text
\usepackage{csquotes}
\usepackage{stackengine}
\usepackage{graphicx}
\usepackage{siunitx}
\def\asterism{\par\vspace{1em}{\centering\scalebox{1.5}{%\\
 \stackon[-0.5pt]{\bfseries*~*}{\bfseries*}}\par}\vspace{.5em}\par}

\usepackage{yfonts,color}
\usepackage{mystyle}
\usepackage{microtype}

\title{The Lighthouse}
\author{PrestonFarlow}
\date{February 01, 2016}

\hypersetup{
 pdfauthor   = {PrestonFarlow},
 pdfkeywords = {Reddit, nosleep},
 pdftitle    = {The Lighthouse}
}

%%%%%%%%%%%%%%%%%%%%%%%%%%%%%%%%%%%%%%%%%%%%%%%%%%%%%%%%%%%%%%%%%%%%%\\
% Begin document                                                    %\\
%%%%%%%%%%%%%%%%%%%%%%%%%%%%%%%%%%%%%%%%%%%%%%%%%%%%%%%%%%%%%%%%%%%%%\\
\begin{document}
\maketitle

% \begin{shadequote}[l]{horriddaydream}
% Well, don't want to live in Pennsylvania anymore.
% \end{shadequote}
% \clearpage

I enjoy the quiet of my island. Most can't stand the idea of living alone, away from everyone and everything, but I've found that it has it's advantages. You can make as much noise as you want, though I rarely even speak. There's far less pressure in how to dress and groom. The sheer amount you're able to think without all the external noise is also impressive. I've read countless books in my time on the island, and I've even taken a stab at writing a few. I could stay awake into the wee hours of the night peacefully typing away at my work, content in the quiet atmosphere around me. Now more than anything I just wish I could return to that time of peace and quiet.\\


Working in a lighthouse can be a stressful job if you lack confidence in your abilities. There's a large assortment of machinery you need to keep in line to make sure everything's running smoothly, and the stakes are high if you fail. You might think that the slow pace of the work would help, but it just gives you more time to worry that you forgot to check something, or that there might be a malfunction somewhere up in the tower. I've been at this job for over a decade now, and I still have these paranoid thoughts. \\


This paranoia helps keep the quality of my work high at least, which is where this story begins in a way. Early one morning I received a radio broadcast from an incoming freighter. They were due to make landfall in the bay sometime before dawn next morning. My lighthouse is on a small island within the bay, about half a mile from the shore. The light's perfectly setup to illuminate the entire coast, and has been helping ships safely dock for over a century now. I'm the latest in a long line of keepers, living a solitary existence on the island, lighting a beacon to guide these ships to safe harbor.\\


But I'm just romanticizing now and should stay on topic. This ship radios in that they'll be arriving some time late that night, and I gave them the general rundown of the harbor. Geography, tidal schedule, open docks, all the relevant information. The weather forecast showed a large storm coming in, but nothing that would stop them from docking. I let them go and made myself a modest breakfast before getting on with my pre-ship inspection. I live in a small cottage with a connecting tunnel to the tower. It's a fairly simple setup, and one that gives me easy access to the light no matter the weather.\\


I ascended to the top of the tower that morning to ensure that everything was in order. The light was bright as ever, the motor spun at the right rate, and the angle was optimal to showing the coastline and any dangerous rock formations. This was all standard procedure that I'd performed hundreds of times before, but constant vigilance is essential for this job. No stone unturned as the saying goes.\\


With the preliminary checks done, I decided to take a brief walk around the island. It's fairly small, only about 5 acres, but I needed the exercise regardless. The morning was grey and calm, as it often is before a large storm. My knee was acting up something fierce, but the walk helped stretch it out. As I was strolling down my beach I looked over the bay, taking in the sights of the city as it slowly awoke. Roustabouts were already hauling various crates and containers off boats, while the neon open signs of various shops and restaurants were flickering to life. It still seems odd, but when I remember that day I always find myself drifting back to this walk. The way the water gently lapped against the jetty, how the wind barely bent the grass, everything had an eerie calm. A proverbial calm before a storm no weather forecast could warn me of.\\


As it was, I finished up my walk and returned to my cabin. The rest of the day passed without event. I checked the light twice more, thoroughness being the key to a good job. When the clouds started to gather and the sky darkened, I went to the tower and flipped its switch. The light shone bright upon the shore, before pivoting off into the horizon of the sea. Waves glittered and shined beneath the illuminating rays and I stood for a moment to marvel. It's a genuine sight to behold, and there's no better view then atop the tower. After that was taken care of, I kept myself busy by reading a novel I'd been working on for quite a while, and keeping tabs on the weather reports. The predictions held steady and rain began to fall around 6 that afternoon. Soon a drizzle turned to a shower, turned to a downpour, turned to a deluge.\\


It got dark quickly with the storm, and I decided to radio the ship. The rain was heavier than expected, and I figured it was best that they know about the conditions. I shot them a quick, hello and waited about half a minute until they hailed back. The signal was very staticy, but I managed to relay the general weather and water conditions. I gave them an advisory to stay out for now, and wait to dock in the morning. Heavy static came through while I waited for a response. Eventually a single sentence reply came through, \enquote{Can't wait\dots  Have to get in\dots  Contact later\dots }\\


I tried to reach them a few more times after this, but didn't get any response. With them refusing to stay out until it was clear, I had no choice but to radio the port authority. Told them that the ship wasn't responding and wasn't changing course. They told me to just make sure the tower is ready to guide them. And I did just that. \\


I took the walk down the hallway and up the stairs to check that everything was in place. The light was spinning at the right rpm, the backup generator had plenty of gas, the radio up top was sending and receiving signals just fine. And just like the last two times I checked, everything was perfectly functioning. I lingered up there for some time, just taking in the sight. The way the light reflected off the rain was stunning, like little drops of gold falling from the sky. The waters itself was equally mesmerizing, the riptide thrashing against the rocks of the island's jetty. I stood there transfixed, for what could have been seconds, for what could have been an hour. All I know is it all fell away when I saw it.\\


My choice of the word 'it' is very deliberate, and something I'll get into later. At that moment, however, what I saw was a man standing alone on my shore, motionless under the pounding rain. Obviously I was far away from him, but even from up in the tower I could tell he was enormous. Pushing seven feet with the massive build to match. And that was about all I could tell as well. No light shone directly on him, and he just appeared as this large, dark mass. \\


I stood there for a moment gaping. Where'd this guy come from? What was he doing on my island? Just who was he? Eventually my senses returned to me and I realized the best way to get these answers would be to ask. While this might sound unwise as I'm telling it now, the context matters. There was a massive storm outside, and I'd seen plenty of ships out in the bay earlier. This seemed like it was someone who'd gotten caught up in the storm and had come ashore trying to escape it.\\


I raced down the steps,back into the cottage, and swung open the front door. The wind almost tore it off it's hinges as it flew open. I hollered outside to the man, only to look around and see that no one was out there. The rain soaked my frontside as I kept yelling after this person for the next several seconds. Eventually the thought occurred to me that he might have run out back. I tried the same thing at the back door in my kitchen, only to be greeted by more rain and no man. Confused and nervous, I closed the door and went back into my study. At this point I began feeling unnerved. The idea of how eerie the situation was began to sink in and I caught myself peering out the window looking for this man. I swore I'd seen someone when I was in the tower, but there was no one on the island. Had it just been my mind playing tricks on me?\\


Try as I might, I couldn't shake the feeling that something was outside. Eventually this sense of dread became too great, and I grabbed my baseball bat. I set it on my lap and tried to continue reading my book, in the hopes of calming my nerves. It would be another hour until the ship was in harbor, and I was right by the main radio in case they called in an emergency. The book proved unhelpful at settling me down. I found myself reading the same page over and over, but not taking in any information. Lightning began to strike outside, and I almost jumped out of my skin at the first crack of thunder. Time passed by painfully slow, with each minute dragging for a torturously long period. I was about to lose it, I needed the radio to buzz or a fog horn to blare, something to break this.\\


Instead all of the lights in the house died.\\


They were quickly revived by the backup generator, but the fright had already knocked me off of my chair. I scrambled backwards into a corner, bat in hand, ready to bludgeon anything that moved. And I waited. After a few minutes I felt comfortable that I was alone. Slowly I began to talk myself back to sanity. It was thundering, of course the power went out. The backup was keeping the light alive, no worries there. Still, I had to check the circuit breaker, make sure nothing was damaged or fried. I picked myself up and walked to the back of the cottage. The fuse box was outside under a small pavilion, next to the meter. The roof protected me from the rain as I opened the door and stepped outside, but I didn't even need to exit the house to see the problem. The main wire leading to the meter and fuses had been crudely and violently severed. And beneath the frayed wires lay several enormous, muddy boot prints.\\


I slammed the door and set both the lock and deadbolt. Turning my back to the wall I scanned the area. No one was in the house, at least not yet. Moving faster then I think I've ever moved before, I began locking every window and door throughout the cottage. Room to room I raced, securing every possible entrance, peering through every bit of glass to try and see who was out there. Someone was on that island with me, and clearly meant me ill. \\


Finally I made it to my study in the back of the house. I locked both windows on the left side and slammed the blinds shut. As I turned to repeat the process on the back window, I froze. Staring at me through that window were two large, green eyes. Unable to move, unable to turn away I found myself lost in them. Adrenaline spiked throughout my body, and I broke into a cold sweat. All I can tell you is that they weren't human. What I saw in those eyes was the hunger and malice of a wild animal.\\


\textit{Boom!}
\\


A flash of lightning blinded me. I frantically blinked my vision back into existence, but even before it was completely back, I saw that the being had moved. Panic set in immediately. Where had it gone? What the was it? What the fuck did it want from? I needed to hide. But where? Where was safe? Not the cottage. I couldn't reach my boat, \textit{it}
 was waiting for me outside. The tower. Metal doors and keypad locks. I'd be safe.
My legs moving before my mind, I set out for the other side of the cottage at a dead sprint. Within seconds I ripped open the door leading into the tunnel. Before I could step through, I heard a crash from the front door of the house. Wood splintered from the hinges and the door bent at the force, but it held. Not waiting for the second attempt, I flew down the tunnel towards the large metal door. I heard the faint echo of the hinges and deadbolt giving way as I punched in the code to the metal door. With one final look back, I saw a large shadow cast across the entrance of the tunnel, before I slammed the door shut.\\


Breathing for the first time in what felt like hours, I tried to steady my heart. I listened to the humming of the backup generator, took large inhales and exhales, and closed my eyes briefly to soothe myself. Before I could become completely calm, my senses flooded back to me. The situation was grim, but I was safe here. Nothing was getting through the towers door. And what's more there was a radio atop the staircase, just below the light. If I could radio ashore, the police might be able to help. I quickly ascended the spiraling stairs, the pounding of the rain outside echoing my footsteps against the steel steps. The light above me spun at it's almost hypnotic rhythm, momentarily consumed in the flashes of lightning outside.\\


Before I had reached the top I could hear the buzzing being emitted from the radio. Static and screaming. Hastening my pace, I summited the tower and raced towards the receiver and speaker. Multiple voices were coming in over the radio. I tried to focus in and eventually  I recognized two distinct voices. One was a friend of mine from the port authority, the other was the man aboard the freighter I had previously talked to.\\


\enquote{\textit{bzzzzzt}
\dots  Please send help! We're dy- \textit{bzzt}
 - out here!}\\


\enquote{\textit{bzzzzzt}
\dots  Please repeat. You're signals barely getting through.}\\


\enquote{\textit{bzzzzzzzzzzzzzzzzzzt}
\dots  THEY'RE ABOARD THE FUCKING SHIP! WE NE -\textit{bzzt}
- ELP! PLEASE, SOMEBOD-}\\


\textit{Bang!}
\\


That wasn't the radio. It was from downstairs. That thing was pounding on the door. Soon a second thump rang, accompanied by a third seconds later. I stood frozen, lost in the cacophony surrounding me. The banging on the door, the screaming from the radio, the pounding of the rain, the thrumming of the engine, the only thing missing was-\\


\textbf{BOOM!}
\\


I'm still not quite sure where the lightning had hit the tower, all I'm sure of is that it struck me deaf, blind, and prone. I didn't even react to catch my fall, instead letting my head bounce off the metal grating. It's possible I could have blacked out for a moment and if God had any mercy, perhaps he would have let it last longer. Instead I slowly awoke, disoriented and damp from sweat. My head throbbed with both the pain of my fall and the ringing in my ears. As I opened my eyes, I saw a world gone opaque, a white blur over my field of vision. Slowly I breathed in and out, attempting to steady my heart and regain my senses. The white began to fade from my vision, with only the blur and spinning remaining. The ring in my ears began to fade and soon I began to hear noises over it. First and foremost was the falling rain outside. The rhythmic pitter patter of the water droplets against the metal roof, growing louder and louder, drawing closer and\dots \\


It wasn't rain I was hearing. No, it was too loud, too uniform, too close. It was footsteps, and they were ascending the stairs. Desperately I attempt to stand, but my arms gave way before I was even on my knees. I struggled to crawl across the grating,tears forming in the corners of my eyes and streaming down my face. And all the while the pounding continued, drawing ever closer to me.\\


I was faced away when it finally reached the top of the stairs. I don't think I would have faced it even if I could. The fear was too great and I just lay there on my stomach, gently sobbing. Soon I saw one single boot slam down barely an inch from my face. I craned my head upwards to look upon my tormentor, wanting nothing more than to look away, but being unable to stop myself. There I saw the eyes again, in all there emerald, primal fury. The last thing I remember was the flash of teeth beneath this things grinning, torturous eyes, as it let out a loud, unbearable hiss and swiftly connected it's boot with my face.\\


I awoke in the hospital the following day. The police had been alerted by the port authority that something had happened and they later found me unconscious at the top of the tower. Apparently both the cottage and tower had been completely torn apart. Windows broken, doors ripped off their hinges, the light and radio both smashed to bits. I somehow managed to escape with a simple broken nose and mild concussion. There was an officer waiting in my room when I awoke, hoping to get a statement. I relayed to him my story, just as I'm doing to you know. He seemed incredulous at certain details, but didn't voice anything aloud.\\


I asked what had happened with the freighter. According to reports, they had radioed in distress signals, saying that several things had come aboard. Not people mind you, but things. They apparently grew more panicked as they sent out the signal, and soon after the light was smashed the ship had run aground. The Coast Guard was dispatched to help the crew members, but when they arrived they found most had vanished, save a few mutilated corpses. Several pieces of cargo were also missing, but according to official documents these were simply \enquote{Personal Items}.\\


I recovered quickly and was discharged the next day. Instead of returning to the island I opted to stay with a friend on the mainland. I've been in contact with the police several times, but they don't seem to be taking me seriously. Truth be told, I'm having a hard time taking myself seriously.  All I know is that this story is the second thing I've written today. The first was my letter of resignation from the lighthouse. I have family in Iowa that I'm sure will be happy to see me.\\
\end{document}